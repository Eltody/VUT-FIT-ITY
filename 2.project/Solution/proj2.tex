\documentclass[a4paper, 11pt, twocolumn]{article}
\usepackage[left=1.5cm, top=2.5cm, text={18cm, 25cm}]{geometry}
\usepackage[czech]{babel}
\usepackage[utf8]{inputenc}
\usepackage{times}
\usepackage[IL2]{fontenc}
\usepackage{amsthm}
\theoremstyle{plain}
\newtheorem{definition}{Definice}
\usepackage{amsfonts} 
\usepackage{amsmath}

\theoremstyle{plain}
\newtheorem{sentence}{Věta}




\begin{document}

    \begin{titlepage}
        \begin{center}
            \Huge
            \textsc{Vysoké učení technické v Brně} \\
            \huge
            \textsc{Fakulta informačních technologií} \\[0,4em]
            \vspace{\stretch{0.382}}
            \LARGE Typografie a publikování\,--\,2. projekt \\
            Sazba dokumentů a matematických výrazů\\[0,3em] 
            \vspace{\stretch{0.618}}
        \end{center}
        {\Large 2022 \hfill Tomáš Zaťko (xzatko02)}
    \end{titlepage}
    
    \section*{Úvod}
    
    V této úloze si vyzkoušíme sazbu titulní strany, {ma\-te\-matic\-kých} vzorců, prostředí a dalších textových struktur obvyklých pro technicky zaměřené texty (například rovnice (\ref{eq_2}) nebo Definice \ref{abecedaOmega} na straně \pageref{abecedaOmega}). Pro vytvoření těchto odkazů používáme příkazy \verb|\label|, \verb|\ref| a \verb|\pageref|.
    
    Na titulní straně je využito sázení nadpisu podle {op\-tic\-ké\-ho} středu s využitím zlatého řezu. Tento postup byl probírán na přednášce. Dále je na titulní straně použito odřádkování se zadanou relativní velikostí 0,4 em a 0,3~em.
    
    \section{Matematický text}

    Nejprve se podíváme na sázení matematických symbolů a~výrazů v plynulém textu včetně sazby definic a~vět s~využitím balíku \texttt{amsthm}. Rovněž použijeme poznámku pod čarou s použitím příkazu \verb|\footnote|. Někdy je vhodné použít konstrukci \verb|${}$| nebo \verb|\mbox{}|, která říká, že (matematický) text nemá být zalomen. 
    
    \begin{definition}
        \label{NTS}
        {\normalfont Nedeterministický Turingův stroj} (NTS) je šestice tvaru $ M =  (Q, \Sigma, \Gamma, \delta, q_0, q_F)$, kde:
        
        \begin{itemize}
        \item $Q$ je konečná množina {\normalfont vnitřních (řídicích) stavů},
        \item $\Sigma$ je konečná množina symbolů nazývaná {\normalfont vstupní abeceda}, $\Delta \not\in \Sigma$,
        \item $\Gamma$ je konečná množina symbolů, $\Sigma \subset \Gamma$, $\Delta \in \Gamma$, nazývaná {\normalfont pásková abeceda},
        \item $\delta : (Q$ $\setminus$ \{$q_F \}) \times \Gamma \rightarrow 2^{Q\times(\Gamma\cup\{L,R\})}$, kde $L, R \not\in \Gamma$, je parciální {\normalfont přechodová funkce}, a
        \item $q_0 \in Q$ je {\normalfont počáteční stav} a $q_f \in Q$ je {\normalfont koncový stav}.
        \end{itemize}
        
        {\normalfont Symbol $\Delta$ značí tzv.} blank {\normalfont (prázdný symbol), který se vyskytuje na místech pásky, která nebyla ještě použita.}
        
        Konfigurace pásky {\normalfont se skládá z nekonečného řetězce, který reprezentuje obsah pásky, a pozice hlavy na tomto řetězci. Jedná se o prvek množiny} \{$\gamma \Delta^\omega$ \textbar \;$\gamma \in \Gamma^{\ast}$\} $\times$ $\mathbb{N}$\footnote{Pro libovolnou abecedu $\Sigma$ je $\Sigma^\omega$ množina všech \emph{nekonečných} řetězců nad $\Sigma$, tj. nekonečných posloupností symbolů ze $\Sigma$. }.
        Konfiguraci pásky {\normalfont obvykle zapisujeme jako $\Delta xyz$\underline{$z$}$x\Delta...$ (podtržení značí pozici hlavy).} Konfigurace stroje {\normalfont je pak dána stavem řízení a konfigurací pásky. Formálně se jedná o prvek množiny $Q \times \{ \gamma \Delta^\omega$ \textbar $\;\gamma \in \Gamma^{\ast}$\} $\times$ $\mathbb{N}$}.
    \end{definition}
    
    \subsection{Podsekce obsahující definici a větu}
    \begin{definition}
        \label{abecedaOmega}
        {\normalfont Řetězec $w$ nad abecedou $\Sigma$ je přijat NTS~$M$}, jestliže $M$ při aktivaci z počáteční konfigurace pásky $\underline{\Delta}w\Delta...$ a počátečního stavu $q_0$ může zastavit přechodem do koncového stavu $q_F$,  tj. $(q_0, \Delta w\Delta^\omega, 0) \underset{M}{\overset{*}{\vdash}} (q_F, \gamma, n)$ pro nějaké $\gamma \in \Gamma^\ast$ a $n \in \mathbb{N}$.
        
        Množinu $\;L(M)\ = $\; \{$w\ \ $\textbar $\ \ $ $w$ je přijat NTS $M$\} $\; \subseteq \; \Sigma^\ast$ nazýváme {\normalfont jazyk přijímaný NTS} $M$.
    \end{definition}
    
    Nyní si vyzkoušíme sazbu vět a důkazů opět s použitím balíku \texttt{amsthm}.
    \begin{sentence}
        Třída jazyků, které jsou přijímány NTS, odpovídá {\normalfont rekurzivně vyčíslitelným jazykům}.
    \end{sentence}
    
    \section{Rovnice}
    Složitější matematické formulace sázíme mimo plynulý text. Lze umístit několik výrazů na jeden řádek, ale pak je třeba tyto vhodně oddělit, například příkazem \verb|\quad|.

    $$x^2-\sqrt[4]{y_1*y_2^3}\quad x > y_1 \geq y_2\quad z_{z_z} \neq \alpha_1^{\alpha_2^{\alpha_3}}$$
    
    V rovnici (\ref{eq_1}) jsou využity tři typy závorek s různou explicitně definovanou velikostí.
    
    \begin{eqnarray}
		\label{eq_1} 
		x & = & \bigg\{a \oplus \Big[b \cdot \big(c \ominus d \big) \Big] \bigg\}^{4/2} \\ 
		\label{eq_2} 
		y & = & \lim_{\beta\to\infty} \frac{\tan^2\beta - \sin^3\beta}{\frac{1}{\frac{1}{\log_{42}x}+\frac{1}{2}}}
	\end{eqnarray}
	
	V této větě vidíme, jak vypadá implicitní vysázení limity $\lim_{n \to \infty} f(n)$ v normálním odstavci textu. Podobně je to i s dalšími symboly jako $\bigcup_{N\in\mathcal{M}}N$ či $\sum_{j=0}^n x_j^2$. S vynucením méně úsporné sazby příkazem \verb|\limits| budou vzorce vysázeny v podobě $ \lim\limits_{n\to\infty} f(n)$ a $\sum\limits_{j=0}^n x_j^2$.

    \section{Matice}
    Pro sázení matic se velmi často používá prostředí \texttt{array} a závorky (\verb|\left|, \verb|\right|).
    \begin{flushleft}
    \bigskip
    $
		\mathbf{A} =
		\left|
		\begin{array}{cccc}
			a_{11} & a_{12} & \ldots & a_{1n} \\
			a_{21} & a_{22} & \ldots & a_{2n} \\
			\vdots & \vdots & \ddots & \vdots \\
			a_{m1} & a_{m2} & \ldots & a_{mn}
		\end{array}
		\right|=
		\left|
		\begin{array}{cc}
			t & u \\
			v & w
		\end{array}
		\right|= tw - uv
	$
	\end{flushleft}
	
	Prostředí \texttt{array} lze úspěšně využít i jinde.
	
	$$
		\binom{n}{k} =
		\left\{
		\begin{array}{ll}
			\frac{n!}{k!(n - k)!} & \text{pro } 0 \leq k \leq n \\ \quad\; 0 & \text{pro } k > n \text{ nebo } k < 0
		\end{array}
		\right.
	$$
	
\end{document}
